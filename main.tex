\documentclass[11pt,a4paper]{article}
\usepackage[utf8]{inputenc}
\usepackage[T1]{fontenc}
\usepackage{geometry}
\usepackage{xcolor}
\usepackage{enumitem}
\usepackage{graphicx}
\usepackage{array}
\usepackage{parskip}
\usepackage{eurosym}
\usepackage{multicol}
\raggedbottom

% Marges personnalisées
\geometry{
  left=1.5cm,
  right=1.5cm,
  top=1cm,
  bottom=1.5cm
}

% Définition des couleurs
\definecolor{accentcolor}{RGB}{0,90,160}
\definecolor{graytext}{RGB}{100,100,100}

% Personnalisation des listes
\setlist[itemize]{
  leftmargin=*,
  nosep,
  topsep=0pt,
  partopsep=0pt,
  before=\vspace{0.2em},
  after=\vspace{0.6em}
}

% Commande pour titre de section
\newcommand{\cvsection}[1]{%
  \vspace{0pt}
  \noindent
  \textcolor{accentcolor}{\rule{2cm}{1.5pt}}\hspace{0.5em}
  {\large\bfseries\color{accentcolor}#1}
  \vspace{0.5em}
  \par\nobreak
}

% Environnement pour les expériences
\newenvironment{experience}[2]{%
  \vspace{0.1em}
  \noindent\textbf{\color{accentcolor}#1}\hfill\textcolor{graytext}{#2}
  \vspace{-0.1em}
  \begin{itemize}
}{%
  \end{itemize}
  \vspace{0.1em}
}

\begin{document}

% Entête du CV : deux colonnes séparées proprement
\begin{minipage}[t]{0.8\textwidth}
    \vspace*{0.8cm} % Remonte SEULEMENT le titre
    {\Large\bfseries\color{accentcolor} Chargée de mission digitale\\(Ouverte à divers métiers)} \\ \hfill Alternance Winside – Transformation Numérique des Entreprises (TNE)\\ NoCode, LowCode, Power Platform \& Automatisation\\[0.4em]
    {\small\itshape Sept. 2025 - 4 jours en entreprise/
1 jour en formation}
\end{minipage}%
\hfill
\begin{minipage}[t]{0.2\textwidth}
    \raggedleft
    \vspace*{-0cm} % Remonte SEULEMENT la photo
    \includegraphics[width=2.5cm]{nouss.jpg}\\[0.5em]
    {\normalsize\bfseries\color{accentcolor}Noussaiba AYADI}
\end{minipage}

\vspace*{0.0cm}  % <-- ajuste ici pour remonter la section Contact
\begin{minipage}[t]{0.6\textwidth}
  \cvsection{Contact}
  {\large
  \begin{itemize}
    \item \textbf{Email:} \texttt{noussaiba.ayadi@gmail.com}
    \item \textbf{Tél:} +33 6 58 92 84 94
    \item \textbf{LinkedIn:} \texttt{linkedin.com/in/noussaiba-ayadi}
    \item \textbf{Adresse:} 92330, Sceaux (Mobilité IDF)
  \end{itemize}
  }
\end{minipage}

\cvsection{Profil}
\begin{itemize}
    \item ingénieur avec un niveau bac +8, avec plus de 10 ans d'exprience  expérimentée en gestion de projet, qualité et coordination interdisciplinaire.
       \item En spécialisation sur les méthodes et outils \textbf{NoCode, LowCode et d’automatisation}, avec un focus sur la \textbf{Power Platform} (Power Apps, Power Automate, AI Builder).
       \item Ouverte à différents types de missions :
    \begin{itemize}
        \item Chargée de mission digitale
        \item Coordinatrice de projets digitaux
        \item Assistante transformation numérique
        \item Testeuse fonctionnelle / QA / Testeuse NoCode
        \item Data coordinator
        \item Product owner junior
        \item Chargée d'automatisation de processus
        \item Support outils digitaux / référente outils collaboratifs
        \item Conceptrice d'applications NoCode (Power Apps / Glide / Softr)
        \item Assistante chef de projet digital
    \end{itemize}
  \end{itemize}

\cvsection{Compétences clés – Coordination et outils digitaux}
\begin{itemize}
    \item \textbf{Création d'applications NoCode} : développement d'outils et d’interfaces avec Power Apps, Notion ou Airtable adaptés aux besoins métier.
    
    \item \textbf{Automatisation de processus} : conception de flux automatisés via Power Automate et Zapier pour optimiser les tâches répétitives (RH, administratives, commerciales…).
    
    \item \textbf{Utilisation de la Power Platform} : intégration de Power Apps, Power Automate, Copilot Studio et AI Builder pour créer des solutions digitales intelligentes et connectées.
    
    \item \textbf{Test fonctionnel et qualité logicielle} : rédaction de cas de test, exécution de tests manuels et automatisés, validation de fonctionnalités et remontée des anomalies.
    
    \item \textbf{Structuration et exploitation de bases de données} : collecte, centralisation et mise en forme des données pour un usage métier (Excel, Power BI, Notion, SharePoint).
    
    \item \textbf{Connaissance du marketing digital} : maîtrise des principes de visibilité en ligne, parcours utilisateur, et outils de communication digitale (emailing, CRM, réseaux sociaux).
    
    \item \textbf{Gestion de projet digital} : suivi des livrables, animation d’équipe projet, coordination avec les parties prenantes, reporting d’avancement (Jira, Trello, MS Project).
    
    \item \textbf{Collaboration à distance et support aux équipes} : usage avancé d’outils collaboratifs (\textit{Slack, Zoom, Miro, Notion}) pour assurer la communication, le support technique ou fonctionnel, et le travail en équipe hybride.
    
    \item \textbf{Méthodologie orientée projet} : pratique de la pédagogie par projet, résolution de problématiques métiers réelles, prototypage de solutions, et déploiement opérationnel.
\end{itemize}
\newpage
\cvsection{Expériences Professionnelles}

\begin{experience}{Ingénieure Assurance Qualité Logicielle | IT Phare Production}{2022--2025}
  \item Pilotage de l’assurance qualité et de la performance d’une plateforme e-commerce B2B dans un environnement agile et international (Tunisie, Italie, Espagne)
  \item Gestion de projets digitaux intégrant CRM, ERP et e-commerce, avec automatisation via \textbf{java}, \textbf{Python}
  \item Analyse de données commerciales, suivi de \textbf{KPIs} et production de \textbf{reportings} avec \textbf{Excel} et \textbf{Power BI}
  \item Amélioration de la \textbf{base de données client} : nettoyage, structuration, enrichissement, fiabilisation des données
  \item Suivi de l'intégration continue via \textbf{Jenkins (CI/CD)}, gestion des livrables avec \textbf{Jira}, documentation dans \textbf{Confluence}
  \item Animation de réunions hebdomadaires en anglais, coordination avec les équipes projet et parties prenantes
  \item Utilisation quotidienne d’\textbf{outils digitaux} : \textbf{Slack}, \textbf{Zoom}, \textbf{Trello}, automatisation de tâches récurrentes
\end{experience}


\begin{experience}{Cheffe de Projet | I-Services \& Consulting}{2021--2022}
  \item Contexte : Pilotage administratif de projets IT/BTP (jusqu’à 150M€) pour le Groupe Chimique Tunisien
  \item Coordination d’équipes pluridisciplinaires (50+ collaborateurs) dans un environnement multi-acteurs et interfonctionnel
  \item Suivi des livrables (pilotage opérationnel, respect des délais, budgets et exigences qualité)
  \item Gestion administratif avec les parties externes (banques, assurances, sociétés de recouvrement)
 
\end{experience}

% Coupure pour équilibrer les deux pages
 \begin{experience}{Ingénieure Qualification \& Qualité | Industriel de traitement (Tanneries de Sud)}{2019--2021}
  \item Contexte : Projet de mise en conformité environnementale (site industriel 20 employés) face à des non-conformités critiques (chrome, sulfures)
  \item Structuration et déploiement de protocoles IQ/OQ/PQ pour l’installation de nouveaux équipements de traitement
  \item Mise en œuvre d’actions correctives ayant permis une réduction de 85\% des polluants dans les effluents
  \item Conformité réglementaire et obtention de la certification ISO 14001, et la levée des risques de pénalité
  \end{experience}
\begin{experience}{Expériences antérieures}{2004--2019}

\item \textbf{Consultante en Innovation Scientifique} -- Laboratoire de Chimie de Coordination (2013--2016)  
Projet de R\&D sur la modélisation de catalyseurs pour des applications durables. Réalisation de 500+ composés avec un rendement de 65--85\%. Rédaction de documentation scientifique structurée.

\item \textbf{Ingénieure R\&D Nucléaire} -- Centre National des Sciences Nucléaires (2010--2012)  
Optimisation de procédés de stérilisation médicale, développement d’un biocapteur gamma avec une précision de 95\%. Réduction de 50\% des incidents. Publication scientifique chez Springer.

\item \textbf{Responsable Technico-Commercial} -- Coates Lorilleux (2004--2008)  
Pilotage qualité et conformité des encres pour emballages alimentaires (normes FDA 21 CFR). Développement d’encres UV écoresponsables, réduction des coûts de production de 15\%.

\end{experience}
\cvsection{Diplôme \& Formation}
\begin{itemize}
    \item \textbf{Diplôme d’Ingénieure QA Logicielle} -- EPSI, École d’ingénierie informatique, Paris (2023--2024)\\
    Spécialisation : Analyste Fullstack -- Automatisation des tests, DevOps

    \item \textbf{Doctorat en Sciences Industrielles} -- INSAT, Institut National des Sciences Appliquées et de Technologie, Tunis (2019)

    \item \textbf{Certifications} : ISTQB Foundation (2023) -- PSM I (Scrum.org, 2022) -- TOEIC Niveau B2
\end{itemize}


\cvsection{Langues}
\begin{itemize}
  \item Français : Bilingue / Anglais : Niveau B2 (TOEIC)/ Arabe : Langue maternelle
\end{itemize}

\end{document}